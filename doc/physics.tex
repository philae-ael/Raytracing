\section{The physics behind the images}

\subsection{Electronic waves, light transport equation}
\begin{definition}[Electromagnetic radiation]
    \label{def:em_rad}
    It consists of waves of EM which propagate through space
    It's behaviour is determined by the Maxwell equations:
    \begin{itemize}
        \item \(\dive \vec E = \frac{\rho}{\epsilon_0}\),
        \item \(\rot\vec E = -\frac{\partial \vec B}{\partial t}\),
        \item \(\dive \vec B = 0\),
        \item \(\rot \vec B = \mu_0 \left(\vec j + \epsilon_0 \frac{\partial \vec E}{\partial t}\right)\).
    \end{itemize}
    Where \(\vec j\) is the current density, \(\rho\) is the charge density.
\end{definition}


\begin{theorem}[Light transport equation]
    Every solution to Maxwell equation respect the following transport equation
    \begin{itemize}
        \item \(\Delta \vec E = \frac{1}{c_0^2} \frac{\partial^2 \vec E}{\partial t^2}\)
        \item \(\Delta \vec B = \frac{1}{c_0^2} \frac{\partial^2 \vec B}{\partial t^2}\)
    \end{itemize}

    where \(c_0^2 = \mu_0\epsilon_0\) and \(\Delta\) is the Laplacian.
\end{theorem}

\subsection{Poynting vector, Radiance, Radiant energy\ldots}
\begin{definition}[Poynting vector]
    The Poynting vector is the vector defined by:
    \[
        \vec\Pi = \frac{\vec E \wedge \vec B}{\mu_0}.
    \]
\end{definition}

\begin{theorem}[Poynting theorem]
    \[
        -\frac{\partial u}{\partial t} = \dive \vec \Pi + \vec j \cdot \vec E
    \]
    Where \(u = \frac{1}{2}\left(\epsilon_0 E^2 + \frac{B^2}{\mu_0}\right)\) is the local electromagnetic energy, \(j\) is the current density.
\end{theorem}

\begin{definition}[Radiant Energy]
    It is the energy of an electromagnetic radiation, given by
    The radiant energy flux is given by
    \[\Phi_e = \frac{\mathrm d Q_e}{\mathrm{d} t}\]
    where
    \[
        Q_e = \int_\Sigma \vec \Pi \cdot \vec n\ \mathrm d A
    \]
    and \(\Sigma\) is a closed surface.
\end{definition}
\begin{theorem}
    \[
        \Phi_e \approx \int_\Sigma \left< |\Pi| \right> \cos \alpha\ \mathrm d A
    \]
    Where \(\alpha\) is the angle between \(\vec \Pi\) and \(\vec n\).
\end{theorem}

\begin{definition}[Radiance]
    The radiance is the radiant flux emitted, reflected, transmitted or received by a given surface, per unit solid angle per unit projected area
    \[
        L_{e, \Omega} = \frac{\partial^2 \Phi_e}{\partial \Omega\,\partial (A \cos \theta)},
    \]
    where \(\Omega\) is the solid angle, \(A \cos \theta\) the projected area.

\end{definition}

\subsection{The rendering equation}
\begin{definition}
    The rendering equation is an equation that gives the relation between the outgoing radiance along an output direction \(\omega_o\) at a given point,
    and the incoming radiance:
    \[
        L_o(\vec x, \omega_o, t) = L_e(\vec x, \omega_o, t) + \int_\Omega f_r(\vec x, \omega_i, \omega_o, t)L_i (\vec x, \omega_i, \omega_o, t) (\omega_i \cdot \vec n) \,\mathrm d \omega_i.
    \]
    where \(L_o\) is the total outgoind radiance along \(\omega_0\),
    \(\vec x\) the position, \(t\) the time, \(L_e\) the emitted radiance, \(L_i\) the incoming radiance, \(f_r\) is the BRDF function.

    The equation can also be modified to take into account the wavelength
\end{definition}

The previous equation has some limitations: it can't model non-linear effects, interference, polarization...

See \citedef{def:transient_rendering}{transient rendering} and \citedef{def:volume_rendering}{volume rendering}.

\begin{definition}[Bidirectional reflectance distribution function --- BRDF]\label{def:BRDF}


\end{definition}